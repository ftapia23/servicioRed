%%%%%%%%%%%%%%%%%%%%%%%%%%%%%%%%%%%%%%%%%%%%%%%%
% Título:	  Documento de análisis Servidor DHCP
%Elaboró: TAPIA MUJICA FERNANDO
%%%%%%%%%%%%%%%%%%%%%%%%%%%%%%%%%%%%%%%%%%%%%%%%

\documentclass{article}
\usepackage{graphicx}
\usepackage[utf8]{inputenc}
\usepackage[spanish]{babel}
\usepackage{color}

\definecolor{gray97}{gray}{.97}
\definecolor{gray75}{gray}{.75}
\definecolor{gray45}{gray}{.50}
\definecolor{arsenic}{rgb}{0.23, 0.27, 0.29}
\definecolor{airforceblue}{rgb}{0.36, 0.54, 0.66}
\definecolor{aliceblue}{rgb}{0.94, 0.97, 1.0}
\definecolor{babyblueeyes}{rgb}{0.63, 0.79, 0.95}
\definecolor{ballblue}{rgb}{0.13, 0.67, 0.8}

\usepackage[x11names,table]{xcolor}
\usepackage[T1]{fontenc}

\usepackage{amsmath, amsthm, amsfonts}
%%%%%%%%
%%\renewcommand\tocbibname{Referencias}
\usepackage[nottoc]{tocbibind}
%%%%%%%%

%%%%%%%%
%%Libreria de colores, y definición de nuevos
\usepackage{color}
\definecolor{grisClaro}{gray}{0.95}
\definecolor{naranja}{rgb}{1,0.5,0}
%%%%%%%%

%%%%%%%
%%Item de colores (Azul)
\usepackage{tikz}
\usepackage{enumitem}
\newcommand*{\itembolasazules}[1]{%
\footnotesize\protect\tikz[baseline=-3pt]%
\protect\node[scale=.5, circle, shade, ball color=blue]{\color{white}\Large\bf#1};}
%%%%%%%

%%%%%%%
%%Modificar Margenes
\usepackage{anysize}
\marginsize{2.5cm}{2.5cm}{2cm}{2.5cm}
%%%%%%%%

\usepackage{listings}

% minimizar fragmentado de listados
% Fracción para elementos que semejan comandos en consola.
\lstnewenvironment{listing}[1][]
{\lstset{#1}\pagebreak[0]}{\pagebreak[0]}

\lstdefinestyle{consola}
{basicstyle=\scriptsize\bf\ttfamily,
backgroundcolor=\color{gray75},
}

\lstdefinestyle{texto}
{basicstyle=\scriptsize\bf\ttfamily,
backgroundcolor=\color{babyblueeyes},
}
%%%%%%%%%%%%

\usepackage{titlesec}

\titleformat{\section}
{\color{ballblue}\normalfont\Large\bfseries}
{\color{ballblue}\thesection.}{1em}{}

\begin{document}
\title{Transportes Jaguar}
\author{Matrices estratégicas\\
\small Gestión empresarial\\
\small México, D.F.
}

\vspace{2cm}
\maketitle		
	\vspace*{-1cm}
	\begin{center}\rule{0.9\textwidth}{0.1mm}\end{center}
	\begin{abstract}
	\normalsize Transportes Jaguar busca representar los factores ó características más importantes dentro 
	y fuera de la empresa, para tomar medidas de mejora y así  seleccionar el tipo de estrategia más apropiada 
	a la institución en función a los objetivos que se han venido buscando.
	\begin{center}\rule{0.9\textwidth}{0.1mm}\end{center}
	\vspace*{0.5cm}
	\end{abstract}
	\vspace*{9cm}
\small Autor(es)
\begin{itemize}
	\item Tapia Mujica Fernando
	\item  Daniel Eduado Aguilar Zúñiga
\end{itemize}
	\tableofcontents
\section{Análisis estratégico de la empresa}
	Es tiempo de dar a conocer factores, características ó  parámetros importantes tanto benéficos como malignos para en nuestra empresa, los cuales nos han mostrado el avance y retroceso en estos 5 años, para Transportes Jaguar \textbf{(TJ)} es de suma importancia la evaluación constante con el fin de actuar de manera preventiva o definitiva ante el ambiente en el que nos encontramos y desarrollamos. Con ello se muestran los siguientes factores de estudio:
	
\begin{itemize}
	\item[\itembolasazules{F}] Fortalezas
		\begin{enumerate}
			\item El precio de nuestras ventas no aumentan en mayor grado que nuestro volumen de ventas en los 5 años transcurridos, el aumento por periodo es de 10\% en relación a los anteriores.
			\item Las unidades con las que se cuenta son de última generación provocando así la confianza del cliente (Equipo nuevo), la vida útil de las unidades con las que contamos es de un periodo entre [5-9] años.
			\item (Aumento del 8\% participación de mercado) El compromiso de entrega en tiempo y forma nos da mayor credibilidad en el mercado,  así como relaciones estratégicas con las que contamos por la amplia experiencia de los fundadores en el servicio de auto-transporte nos brinda la certeza de un constante crecimiento.
			\item (Recurso humano) El brindar un servicio de calidad son los puntos más detallistas que tenemos en la organización a consecuencia de ello, todos los recursos que conforman TJ están altamente capacitados  y en constante crecimiento.
			\item (Reducción de la deuda al 0\%) La liquidez con la cual cuenta la empresa al generar recursos constantes nos promueve un apalanca-miento para cumplir con nuestras responsabilidades financieras
			\item Nuestras políticas de calidad e integración de equipo crean una constante competencia de nuestros integrantes.
		\end{enumerate}
	\item[\itembolasazules{D}] Debilidades
		\begin{enumerate}
			\item La principal de ellas es la falta de recursos para solventar la demanda de grandes empresas, por el momento se cuenta con una 
			flota de 5 equipos y 6 operadores.
			\item (Contratación y capacitación plazo extenso para nuevos prospectos) La falta de personal que pueda suplir a los elementos altamente capacitados con los que contamos en el momento que se 	presente algún siniestro(se conoce como siniestro a la perdida de uno de nuestros integrantes).
		\end{enumerate}
	\item[\itembolasazules{O}] Oportunidades
		\begin{enumerate}
			\item Facilidad de crecer por nuestro constante compromiso al cumplir las deudas financieras, por ello se puede buscar una nueva
			inversión para aumentar la flota y así brindar un servicio más  robusto. 
			\item En el distrito federal se observa la poca participación de las grandes empresas del sector de transporte, por ello la pronta actividad de nuestra parte nos ayudara a crecer, (El forjar buenas relaciones y buscar licitaciones que consoliden nuestro servicio con empresas de autoservicio y la central de abastos de la ciudad de México son los sectores de mayor cumulo de negocio).
			\item El equipo de última tecnología con el cual contamos nos da la seguridad de concretar  y sostener de manera satisfactoria contratos con los clientes.
		\end{enumerate}
	\item[\itembolasazules{A}] Amenazas
		\begin{enumerate}
			\item El incremento de precios en los servicios que consumimos (diesel, casetas) provoca el incremento en el costo que nosotros le proponemos al cliente.
			\item El alto índice de inseguridad en las vialidades pueden provocar bajas en nuestras unidades.
			\item El aumento de los impuestos en el sector de trasporte de mercancías.
		\end{enumerate}	 
\end{itemize}

\section{Herramientas de gestión para TJ}

	Después de puntualizar los factores \textit{FODA} se dará seguimiento a implementar herramientas que nos permitan detectar elementos interesantes a partir de la coexistencia de estos al relacionarlos (datos buenos para la empresa contra datos malos).
	\newpage
	\subsection{Matriz de evaluación de factores internos \textbf{(MEFE)}}
\begin{table}[h]
\centering
\begin{tabular}{l|c|c|c}
\hline\hline \rowcolor{LightBlue2} Factor a analizar& Peso relativo &Calificación&Calificación ponderada\\ \hline\hline
\multicolumn{4}{c}{FORTALEZAS}\\ \hline
F-1& 0.2 		& 4& 0.8 \\
F-2 & 0.15 & 4& 0.6 \\
F-3 & 0.1 	& 2&  0.2\\
F-4 & 0.1 	& 2&  0.2\\
F-5 & 0.15 & 3&  0.45\\
F-6 & 0.05 & 2&  0.1\\\hline
\multicolumn{3}{r}{Total}& 2.35\\ \hline
\multicolumn{4}{c}{DEBILIDADES}\\ \hline
D-1& 0.15 & 3& 0.45\\
D-2 & 0.10 & 2& 0.2\\\hline
Total&1 & & 0.65\\ \hline
\end{tabular}
\caption{MEFI}
\end{table}

\subsection{Análisis}
	Como podemos observar las fortalezas muestran un mayor peso en TJ, pero se deben trabajar los elementos negativos, así como fortalecer 
	los puntos positivos los cuales nos están orientando por un camino viable y rentable para la meta que tiene forjada la empresa desde sus inicios.
	El saber que los niveles de ventas aumentan y su costo se mantiene al margen es el claro ejemplo para darle a conocer al mercado nuestro compromiso 	e incesable deseo de cumplir las necesidades de los clientes.
\begin{thebibliography}{99}
\bibitem{c99} R. Droms, \emph{Dynamic Host Configuration Protocol, RFC 2131}, Bucknell University, (March 1997)
\bibitem{c98} R. Droms, S. Alexander, \emph{DHCP Options and BOOTP Vendor Extensions, RFC 2132}, Bucknell University,  (March 1997)
\bibitem{c98} \emph{https://www.isc.org/downloads/dhcp/} and man pages.
\end{thebibliography}

\end{document}
